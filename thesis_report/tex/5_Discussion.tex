\chapter{Discussion:}

The results presented in the previous section confirm the potential of \ac{oae} to lower ocean \ch{pCO2} and enhance the downward \ch{CO2} flux in European waters, as well as the additional benefit of increasing seawater pH and mitigating future ocean acidification. Seasonally, \ac{oae} impacts resulted in the dampening of the ocean \ch{pCO2} monthly cycle and in the amplification of the \ch{CO2} flux cycle, as hypothesised in the first chapter. Lastly, the change in carbon inventory that is observed in \cref{cgt} highlights \ac{oae} as a suitable candidate for achieving the \ac{pa}. Below, an in-depth analysis of the causes that may trigger such system changes are presented. 

\section[\texorpdfstring{OAE}{OAE} impacts at the European average:]{\ac{oae} impacts at the European average:}

All maps show that system perturbations for all variables happen within the model domain, in the proximity of the injection site, and the signal does not propagate far by the end of the century. This leads to conclude that \ac{oae} deployment is spatially limited to where alkalinity is added, at least for the upper 50 metres of the ocean surface. Such outcome is corroborated by a recent experiment carried out by \cite{wang2023simulated}. 

Open-ocean regions like Iceland, Norway and Spain show few-to-absent perturbations with alkalinity addition. The most affected region is the North Sea, and especially the southern North Sea, which delivers pronounced system alterations. In addition to the fact that this is the location where the experiment is simulated, the shallower, well-mixed ocean system allows for alkalinity to accumulate at the surface and to not be lost due to mixing-induced subduction before equilibration is completed \citep{wang2023simulated}. Ocean \ch{pCO2} seasonal amplification at high latitudes diverges from the expected outcome and may be the result of \ch{dic}-driven modulations. 

In SSP1-2.6, both the baseline and the \ac{oae} scenario register higher alkalinity values than their respective SSP3-7.0 levels. This may be the result of multiple mechanisms influencing alkalinity fluxes and corresponding sources and sinks. In the case of \ac{foci}, nutrient consumption and supply or variations of the carbonate counter pump may explain such deviations. 

When observing the seasonal trend of alkalinity in both SSP1-2.6 and SSP3-7.0, a reversed pattern is detected (A and C of \cref{alkalinity}), especially at the very first ocean layer (right graph of \cref{EUalkalinitysurface}). In the baseline, alkalinity values progressively grow with depth. Seasonally, this translates in maxima registered in winter, when stronger winds force surface water mixing with the deeper, more alkalinised subsurface. Minima are then detected in summer, when \ac{sst} increases and enhanced ocean stratification prevents mixing. 

When alkalinity is added to the surface layer, as it is the case of the simulation experiments, a system shift is visible. Alkalinity is high at the top layer and progressively lowers with depth, up to about 100 metres, where an ascending trajectory is restored (left graph of \cref{EUalkalinitysurface}). Winter mixing results in the surface water being incorporated with the less alkalinised subsurface region, whereas summer stratification leaves alkalinity-rich water at the top.

In both SSP1-2.6 and SSP3-7.0, \ch{pCO2} minima are registered in spring, which may be the result of phytoplankton bloom and consequent high photosynthetic activity, where \ch{CO2} is consumed. In SSP3-7.0, major atmospheric \ch{CO2} emissions force \ch{pCO2} to rise progressively until the end of the simulations, even with alkalinity enhancement (not shown). Ocean \ch{pCO2} seasonal amplitude is reduced due to the fact that an alkalinised ocean minimises \ch{pCO2} sensitivity to \ac{dic} changes, described by \ch{CO2} fluxes.

From these results, it is assumed that \ac{rf} lowers in the study region, as seawater buffering capacity is enhanced. This system shift is expressed by the fact that larger \ac{dic} variations are needed to produce the same \ch{pCO2} alterations that would be realised in a non-alkalinised ocean \citep{schwinger2022report}. Larger \ac{dic} inputs are therefore necessary to bring about similar \ch{pCO2} changes. Hence, it is likely \ch{pCO2} sensitivity to its seasonal drivers, rather than the drivers themselves, the cause of this regime shift, as cited in \cite{lerner2021drivers}.

In accordance with previous estimates \citep{schwinger2022report}, \ac{oae} causes an amplification of the \ch{CO2} seasonal flux for both \ac{ssp}s. \ac{oae} enlarges the air-to-sea disequilibrium that requires higher \ch{CO2} input to be restored. In all scenarios, highest levels of alkalinity are recorded between the end of summer and the onset of autumn but largest uptake takes place in winter, therefore introducing a short response lag due to re-equilibration timescales. In the \ac{oae} scenario, alkalinity addition forces a greater \ch{CO2} flux downward in wintertime, meaning that \ac{oae} induces an asymmetrical shift of the ocean regime, with much lower lows than lower highs.

Such stronger \ch{CO2} downward movement in both scenarios takes the form of an increment of the \ac{dic} pool. Highest peaks are in spring and lowest in autumn, consistent with an expected requilibration lag, as greatest uptake happens in winter months. For the same relation expressed above by \ac{rf}, in an alkalinised ocean, larger variations in the \ac{dic} pool elicit the same change in the ocean \ch{pCO2}. However, although maintaining a similar cycle compared to the baseline, \ac{dic} seasonal amplitude is slightly decreased with alkalinity addition, so that relation between alkalinity and \ac{dic} seasonal compression seems to take place (A and B of \cref{alkalinity} and \cref{dic}).  

As predicted, \ac{oae} allows for a pH increase by favouring a shift towards a carbonate-dominated system. With \ac{oae}, ocean acidification is therefore mitigated. Largest baseline-to-\ac{oae} anomaly is recorded in summer, reproducing most pronounced \ac{oae} impacts at the corresponding highest seasonal acidification phase. This translates in a compression of the pH seasonal cycle, also probably due to the fact that \ch{H+} annual concentration grows faster than its seasonal increase \citep{kwiatkowski2022modified, kwiatkowski2018diverging}.

\section[\texorpdfstring{OAE}{OAE} impacts at location S:]{\ac{oae} impacts at location S:}

In the southern North Sea \ch{CO2} flux modulations are dominated by temperature changes (see B of \cref{dsmw}) \citep{rodgers2023seasonal, salt2013variability, prowe2009mechanisms}, resulting in \ac{dic} minima being observed in summer (at highest \ch{pCO2}), when warm temperatures elicit outgassing, and in ocean \ch{pCO2} maxima being registered in winter, at highest gas solubility. Additionally, a much larger flux change is reflected in comparison to the European average. In SSP1-2.6, \ch{CO2} seasonal amplitude doubles and, in SSP3-7.0, it triples, in accordance with \cite{lenton2018assessing} who found greater net uptake under greater atmospheric \ch{CO2} concentration. In both scenarios, the system change is asymmetrical, detecting much lower lows than lower highs (F and H of \cref{co2flux}). 

Air-sea re-equilibration can take from a few months to even years to complete, depending on physical properties such as latitude, air-sea gas exchange, \ac{mld}, water mixing and wind speed \citep{jones2014spatial}. Therefore, drawing from \cite{wang2023simulated} and \cite{jones2014spatial}, it is deduced that the shallower, well-mixed waters of the southern North Sea encourage fast re-equilibration, which trumps mean water residence time and allows for efficient carbon sequestration. Quantifying such temporal gaps between system forcing and air-sea exchange is of paramount importance when addressing \ac{oae} potential, as incomplete requilibration could substantially inhibit \ac{oae} efficiency \citep{wang2023simulated, jones2014spatial}. 

\ch{pCO2} levels plummet, with lowest registered values being 114 µatm at SSP1-2.6 and 277 µatm at SSP3-7.0. Differently from the European average, the thermal component drives seasonality at location S, and minima are registered in winter, at highest seawater solubility. In analogy with \cite{fassbender2018seasonal}, $\Delta$ \ch{pCO2} is highest in the season where the associated seasonal driver operates to reduce the ocean \ch{CO2} sink potential. In this case, $\Delta$ \ch{pCO2} increases in summer, when rising temperatures force a reduced gas solubility and inhibit ocean uptake. \ac{oae} is therefore most efficient during that phase \citep{fassbender2018seasonal}. 

\ac{dic} absolute values are considerably greater than at the European average, reaching 2450 mmol m\textsuperscript{-3} in SSP1-2.6 and 2738 mmol m\textsuperscript{-3} in SSP3-7.0. As expected, the seasonal amplitude is enhanced, especially under the higher-emission scenario, since an alkalinised ocean induces a larger \ac{dic} seasonal cycle to modulate the same \ch{pCO2} response. Larger \ac{dic} variance in SSP3-7.0 means that the ocean buffering capacity is lower, as confirmed by \cite{lenton2018assessing}. This response is likely due to the enhanced seasonal cycle of alkalinity observed in E and G of \cref{alkalinity}, where alkalinity regulates \ac{dic} seasonality, and to higher atmospheric \ch{CO2} concentration, which creates a larger disequilibrium to be restored.  

pH seasonal amplitude remains fundamentally unchanged. Based on previous research, this leads to believe that, with alkalinity enhancement, the annual \ch{H+} concentration rises proportionally to its seasonal fluxes. Under a higher emission scenario, $\Delta$ pH absolute values are lower, meaning that, unlike all other variables where \ac{oae}-induced modulations are highest in SSP3-7.0, alkalinity enhancement is not as effective in bringing about the same mitigating effect to acidification with elevated atmospheric \ch{CO2}. This is corroborated by \cite{lenton2018assessing}, where it was observed that acidification is relieved more efficiently under a lower emission scenario.  
